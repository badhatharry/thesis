\documentclass[12pt]{article}
\usepackage{graphicx}
\usepackage{wrapfig}
\usepackage{subfigure}
\usepackage{multirow}
\usepackage{hyperref}
\usepackage{amsmath}
\usepackage{amssymb}
%\usepackage{ngerman}
\usepackage[ansinew]{inputenc}
\usepackage[left=2cm,top=1cm]{geometry}

\begin{document}
	\pagestyle{empty}


\begin{titlepage}
    \centering
	\huge{Inverse equation of state construction from mass-radius-relations of compact stars}
	\bigskip
    \large{Bachelor thesis}
    \huge{Jan Roeder}
\end{titlepage}


\tableofcontents
\pagebreak


\section{Introduction}

\section{Preparations}

\subsection{The Tolman-Oppenheimer-Volkoff equation}

Before we get to the main topics of this thesis, it is important to set up a basis that our work emerges from. \\
Firstly, this involves a derivation of the equation used for determining the structure of compact stars; here, we will use the 
Tolman-Oppenheimer-Volkoff [TOV] equation. We choose our unit system as $c=G=1$, so that every unit is a power of a length.\\
The derivation is based off assuming the star matter as a perfect/ideal fluid. The system shall further not evolve in time, 
therefore staying spherically symmetric. In terms of the metric components, we are left with;
%insert metric



In conclusion we get the full TOV equation(s):
\begin{equation}\label{eq:tov}
	\frac{dP}{dr} = \frac{(\epsilon + P)(m + 4\pi r^3 P)}{2mr-r^2}
\end{equation}
\begin{equation}\label{eq:mr}
	\frac{dm}{dr} = 4\pi r^2\epsilon
\end{equation}

\subsection{Numerical solution}

In order to generate an initial mass-radius relation to test our reverse algorithm with, we use a fourth order Runge-Kutta
algorithm to solve the TOV equation along with the mass differential equation numerically. 
By looking at both equations, our ODE system is in the form
\begin{equation}\label{eq:ode}
	\dot{y}(t) = f(y(t), t)
\end{equation}
where $\dot{y}(t)$ is a two component "vector":
\begin{equation}
	\dot{y}(t) =  \left( \begin{array}{c}dP/dr\\dm/dr\end{array} \right)
\end{equation}
$f(y(t), t)$ then contains the right hand side of equations~\ref{eq:tov} and~\ref{eq:mr}.
















\pagebreak

%%%ALTERNATIVE BILDUMGEBUNG
% \begin{figure}[h]
% \begin{center}
% \includegraphics[width=50mm]{fig1}
% \caption{Fadenpendel \cite{Eichler}.}\label{fig:1}
% \end{center}
% \end{figure}




\hspace{15pt}

\begin{table}[h!]
  \centering
  %\caption{Caption for the table.}
  \label{tab:2}
  \begin{tabular}{|c||c|}
  \hline
    Messung der Fadenl�nge & $l$ (m)\\
    \hline\hline
    1 &   \\ \hline
    2 &   \\ \hline
    3 &   \\ \hline
    4 &   \\ \hline
    5 &   \\ \hline
    Mittelwert $\bar{l}$ &  \\ \hline
    Standardabweichung $\sigma_l$ &  \\ \hline
  \end{tabular}
\end{table}



\begin{thebibliography}{2}
\bibitem{Eichler} H. J. Eichler, H.-D. Kronfeldt, J. Sahm, \textit{Das Neue Physikalische Grundpraktikum}, Springer-Verlag, Berlin-Heidelberg, 2001.
\bibitem{Kuchling} H. Kuchling, \textit{Taschenbuch der Physik, 21. Auflage}, Fachbuchverlag Leipzig, 2014.
\end{thebibliography}

\end{document}