\documentclass[12pt]{article}
\usepackage{graphicx}
\usepackage{wrapfig}
\usepackage{subfigure}
\usepackage{multirow}
\usepackage{hyperref}
\usepackage{amsmath}
\usepackage{amssymb}
\usepackage{ngerman}
\usepackage[ansinew]{inputenc}
\usepackage[left=2cm,top=1cm]{geometry}

\begin{document}
\pagestyle{empty}
\begin{flushleft}
\begin{table}
\begin{tabular}{lr}
\bfseries\LARGE
Goethe-Universit\"at & \\
\bfseries\LARGE
Frankfurt am Main  & \\\\
\bfseries\LARGE
Physikalisches Institut & \multirow{-6}{0.5\textwidth}{\includegraphics[width=0.5\textwidth]{pilogo}}\\
\bfseries\LARGE
Anf\"angerpraktikum Teil 1 &  \\\\\\
\end{tabular}
\end{table}
\begin{tabular}{lr}
\bfseries\LARGE
Versuchsname: & \\
\\\\
\bfseries\LARGE
Themengebiet: & \\
\\\\
\bfseries\LARGE
Versuchsnummer: & \\
\\\\
\bfseries\LARGE
Studiengang: & \\
\\\\
\bfseries\LARGE
PG-Nummer: & \\
\\\\
\bfseries\LARGE
Praktikanten: & \\
\\\\
\bfseries\LARGE
Assistent: & \\
\\\\
\bfseries\LARGE
Versuchsdurchf\"uhrung am: & \\
\\\\
\bfseries\LARGE
Protokollabgabe am: & \\
\\\\
\bfseries\LARGE
Korrekturabgabe am: & \\
\\\\
\bfseries\LARGE
Versuch bestanden am: & \\
\end{tabular}
\end{flushleft}

\newpage
\section{Aufgabenstellung}
Bestimmen Sie die Erdbeschleunigung $g$ mithilfe eines mathematischen Pendels.

\section{Physikalische Grundlagen}
\begin{wrapfigure}{L}{70mm}
\begin{center}
\includegraphics[width=65mm]{fig1}
\caption{Fadenpendel \cite{Eichler}.}\label{fig:1}
\end{center}
\end{wrapfigure}

%%%ALTERNATIVE BILDUMGEBUNG
% \begin{figure}[h]
% \begin{center}
% \includegraphics[width=50mm]{fig1}
% \caption{Fadenpendel \cite{Eichler}.}\label{fig:1}
% \end{center}
% \end{figure}

Das Mathematische Pendel ist eine Idealisierung des Physikalischen Pendels und
kann nur in N�herung realisiert werden. Es besteht aus einem Massepunkt und
einer masselosen reibungsfreien Aufh�ngung \cite{Eichler}. Eine m�gliche
Realisierung ist ein Fadenpendel mit einem Faden der L�nge $l$ an dem eine
kugelf�rmige Masse $m$ h�ngt (siehe Abb. \ref{fig:1}). Nach einer Auslenkung um den Winkel $\varphi $ vollf�hrt
dieses Pendel Schwingungen um seine Ruhelage. 
Auf den K�rper wirkt die
Gewichtskraft $F = m \cdot g$. Man zerlegt sie in eine Komponente in Richtung des
Fadens (diese wird vom Faden aufgefangen) und in eine Komponente senkrecht dazu
(siehe Abb. \ref{fig:1}). Dies ist die r�cktreibende Kraft $F_R=-m\cdot g \cdot \sin \varphi$.
Sie erzeugt eine Beschleunigung, $a=\ddot{x}=l\cdot\varphi$, mit $\ddot{\varphi}=\frac{\partial^2\varphi}{\partial t^2}$ l�ngs der kreisf�rmigen Pendelbahn. Mit der Newtonschen Grundgleichung  ergibt sich $m\cdot l\cdot \ddot{\varphi}=-m\cdot g\cdot \sin\varphi$ bzw.
\begin{equation}
\ddot{\varphi}+\omega^2\sin\varphi=0 \textnormal{ mit } \omega=\sqrt{g/l}.
\end{equation}
Dabei ist $\omega$ die Kreisfrequenz der Schwingung. F�r kleine Amplituden (kleine Auslenkwinkel mit $\sin\varphi\approx \varphi)$ wird daraus die Schwingungsgleichung eines harmonischen Oszillators mit der Kreisfrequenz $\omega$, die unabh�ngig von der Maximalamplitude $\varphi_0$ ist \cite{Eichler}:
\begin{equation}
\ddot{\varphi}+\omega^2\varphi=0 \textnormal{ und daraus } \varphi(t)=\varphi_0\sin(\omega\cdot t+\alpha) \textnormal{ mit } \alpha \textnormal{...Phasenverschiebung.}
\end{equation}
Damit folgt aus der Schwingungsdauer $T=2\pi/\omega=2\pi\sqrt{l/g}$ direkt die Erdbeschleunigung:
\begin{equation}
\label{Eq:3}
g=\frac{4\pi^2l}{T^2}.
\end{equation}

\newpage
\section{Messwerte}
Die Messung der Periodendauer wird mit der Stoppuhr eines Smartphones durchgef�hrt. Die Anfangs- und Endzeit wird jeweils im Nulldurchgang der Schwingung bestimmt. Die Messung der Fadenl�nge erfolgt mit einem Stahlma�stab. Gemessen wird der senkrechte Abstand zwischen Aufh�ngung der Kugelmasse und Pendelaufh�ngung, $l_{\rm{mess}}$. Um die wirksame Pendell�nge $l$ zu berechnen, wird der Kugelradius, $rl_{\rm{Kugel}}$, und der Abstand der Schraubeneinfassung, $d_{\rm{Schraube}}$, addiert:
\begin{equation}
\label{Eq:4}
l = {l_{{\rm{mess}}}} + {r_{{\rm{Kugel}}}} + {d_{{\rm{Schraube}}}} = {l_{{\rm{mess}}}} + 12.5 \cdot {10^{ - 3}}\,{\rm{m}} + 2.5 \cdot {10^{ - 3}}\,{\rm{m}} = {l_{{\rm{mess}}}} + 0.015\,{\rm{m}}
\end{equation}

Zur Bestimmung des Mittelwertes $(\bar{x})$, der Standardabweichung der Einzelmessung $(\sigma)$ und der Standardabweichung des Mittelwertes $m$ werden folgende Formeln verwendet:
\begin{equation}
\label{Eq:5}
\bar x = \frac{1}{n}\sum\limits_{i = 1}^n {{x_i}}\,\,,\,\,\,\sigma  = \sqrt {\frac{1}{{n - 1}}\sum\limits_{i = 1}^n {{{\left( {{x_i} - \left\langle x \right\rangle } \right)}^2}} }\,\,,\,\,\,m = \frac{\sigma }{{\sqrt n }}
\end{equation}

\begin{table}[h!]
  \centering
  %\caption{Caption for the table.}
  \label{tab:1}
  \begin{tabular}{|c||c|c|}
  \hline
    Messung der Periodendauer & $10\cdot T$ (s) & $T$ (s)\\
    \hline\hline
    1 &  & \\ \hline
    2 &  & \\ \hline
    3 &  & \\ \hline
    4 &  & \\ \hline
    5 &  & \\ \hline
    Mittelwert $\bar{T}$ &  & \\ \hline
    Standardabweichung $\sigma_T$ &  & \\ \hline
  \end{tabular}
\end{table}

\hspace{15pt}

\begin{table}[h!]
  \centering
  %\caption{Caption for the table.}
  \label{tab:2}
  \begin{tabular}{|c||c|}
  \hline
    Messung der Fadenl�nge & $l$ (m)\\
    \hline\hline
    1 &   \\ \hline
    2 &   \\ \hline
    3 &   \\ \hline
    4 &   \\ \hline
    5 &   \\ \hline
    Mittelwert $\bar{l}$ &  \\ \hline
    Standardabweichung $\sigma_l$ &  \\ \hline
  \end{tabular}
\end{table}

\section{Auswertung}
Die Erdbeschleunigung ergibt sich nach Formel \ref{Eq:3} mit dem Mittelwert aus den Messungen f�r die Periodendauer und der Fadenl�nge.

\noindent $\bar{g}=$

\section{Fehlerrechnung}
Der Fehler von $l$ ergibt sich aus dem systematischen Fehler des Ma�stabs $\Delta l_{\rm{sys}}$ und
dem zuf�lligen Fehler $\Delta l_{zuf}$ aus der Messreihe. F�r den Ma�stab gilt: $\Delta l_{\rm{sys}}=5\cdot10^{-4}$~m$ + 5\cdot10^{-4}\cdot \bar{l} $. Der zuf�llige Fehler ergibt sich aus der doppelten Standardabweichung des Mittelwertes

\begin{equation}\label{Eq:6}
\Delta l_{\rm{\rm{zuf}}}=2\cdot\sigma_l=\frac{2\sigma_l}{\sqrt{5}}.
\end{equation}

\begin{equation}\label{Eq:7}
\Delta l=\Delta l_{\rm{sys}}+\Delta l_{\rm{zuf}}=
\end{equation}
Der systematische Fehler der Stoppuhr eines typischen Smartphones liegt bei etwa $\Delta T_{\rm{sys}}=10\cdot10^{-3}$~s$ + 5\cdot10^{-4}\cdot \bar{T}$, der zuf�llige Fehler ergibt sich analog zu $\Delta l_{\rm{\rm{zuf}}}$ :

\begin{equation}\label{Eq:8}
\Delta T=\Delta T_{\rm{sys}}+\Delta T_{\rm{zuf}}=
\end{equation}

Da die beiden Messgr��en  $l$  und $T$  in voneinander unabh�ngigen Messreihen ermittelt wurden, pflanzen sie sich bei der Bestimmung von $g$  nach dem Fehlerfortpflanzungsgesetz von Gauss fort.

\begin{equation}\label{Eq:9}
\Delta g=\sqrt{\bigl{(}\frac{\partial g}{\partial l}\Delta l\bigr{)}^2+\bigl{(}\frac{\partial g}{\partial T}\Delta T\bigr{)}^2}=
\end{equation}
Durch Umformungen erh�lt man den relativen Fehler
\begin{equation}\label{Eq:10}
\frac{\Delta g}{g}=\frac{\Delta g}{\frac{4\pi^2l}{T^2}}=
\end{equation}
Wenn man die Werte in Formel \ref{Eq:9}  und \ref{Eq:10} einsetzt erh�lt man: 

\hspace{15pt}

\noindent 
Absoluter Fehler: $\Delta g=...$ 

\hspace{15pt}

\noindent 
Relativer Fehler: $\frac{\Delta g}{g}=...$

\section{Angabe des Ergebnisses und Diskussion}
In diesem Versuch wurde die Erdbeschleunigung mithilfe eines Fadenpendels bestimmt.
 
\noindent $\underline{\underline{\Delta g=...}}$\\
\hspace{5pt}
\noindent Vergleich mit Literaturwert aus \cite{Kuchling} $g_{Lit}=9.81$~m/s$^2$.
\hspace{5pt}
\noindent Diskussion: enthalten die Messwerte mit Fehlerbereich den Literaturwert, wenn nicht, was sind m�gliche Fehlerquellen, die nicht ber�cksichtigt wurden, etc.  
...\\

\begin{thebibliography}{2}
\bibitem{Eichler} H. J. Eichler, H.-D. Kronfeldt, J. Sahm, \textit{Das Neue Physikalische Grundpraktikum}, Springer-Verlag, Berlin-Heidelberg, 2001.
\bibitem{Kuchling} H. Kuchling, \textit{Taschenbuch der Physik, 21. Auflage}, Fachbuchverlag Leipzig, 2014.
\end{thebibliography}

\end{document}